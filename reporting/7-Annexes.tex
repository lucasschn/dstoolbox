\section{Initialization of the separation function parameters for lest squares optimization}

In order to give meaningful initialization value to the parameters $S_1$ and $S_2$ of Equation \eqref{eq:seppoint} for the optimization process described in section \ref{section:kirchhoff}, it is useful to consider the derivative of the separation point function with respect to $\alpha$, shown below:

\begin{equation} 
\frac{df}{d\alpha} = 
     \begin{cases}
       -\frac{0.3}{S_1}\exp(\frac{\alpha-\alpha_{ss}}{S_1},) &\quad \alpha \leq \alpha_{ss}\\
        -\frac{0.66}{S_2}\exp(\frac{\alpha_{ss}-\alpha}{S_2}) &\quad \alpha > \alpha_{ss}\\
     \end{cases}
\end{equation}

By taking its derivative with respect to the angle of attack

\begin{equation}
\frac{dC_N}{d\alpha} = C_{N_{\alpha}} \left[\left(\frac{1+\sqrt{f}}{2}\right)^2+\frac{f'(\alpha)}{2f}\alpha \right]
\end{equation}

\noindent and taking the left and right limits around the static stall angle,

\begin{equation}
\frac{dC_N}{d\alpha}|_{\alpha=\alpha_{ss}} = C_{N_\alpha}\left(\frac{1+\sqrt{0.7}}{2}\right)
     \begin{cases}
      \left(\frac{1+\sqrt{0.7}}{2}\right)-\frac{0.3\alpha_{ss}}{2\sqrt{0.7}S_1},&\quad \alpha < \alpha_{ss}\\
        \left(\frac{1+\sqrt{0.7}}{2}\right)-\frac{0.66\alpha_{ss}}{2\sqrt{0.7}S_2}, &\quad \alpha > \alpha_{ss}\\
     \end{cases}
\end{equation}

\noindent an expression for $S_1$ and $S_2$ that depends only on the lift curve slope during in the attached regime $C_{N_\alpha}$, the static stall angle $\alpha_{ss}$ and the lift slope around stall is found.

\begin{equation}
S_1 = \frac{0.3\alpha_{ss}}{2\sqrt{0.7}}\left[\left(\frac{1+\sqrt{0.7}}{2}\right)-\frac{1}{C_{N_\alpha}}\frac{dC_N}{d\alpha}|_{\alpha=\alpha_{ss}^{-}}\right]^{-1}
\end{equation}

\begin{equation}
S_2 = \frac{0.66\alpha_{ss}}{2\sqrt{0.7}}\left[\left(\frac{1+\sqrt{0.7}}{2}\right)-\frac{1}{C_{N_\alpha}}\frac{dC_N}{d\alpha}|_{\alpha=\alpha_{ss}^{+}}\right]^{-1}
\end{equation}

\section{The mathematical relationship between lagged AoA time constant and the lagged dynamic stall angle}

As the lagged pitch angle $\alpha'$ is defined as the output of a first order system, which input is $\alpha'$, let's define the transfer function for a first-order, linear time invariant system. 

\begin{equation}
G(s) = \frac{1}{1+\tau s}
\end{equation}

\noindent From control theory, we know that the response in Laplace domain is $Y(s) = G(s)U(s)$, where $Y(s)$ and $U(s)$ are the Laplace transforms of the output $y(t)$ and the input $u(t)$ of the system, respectively.

\subsection{Indicial response}

Let's first inspect the indicial response, in other words, the response to a step input. A step input is $u(t \geq 0) = K$. In the present context, all signals are assumed to be $0$ when time is below 0. In Laplace domain, that is $U(s) = K/s$. The response to a step input is, using partial fraction decomposition : 

\begin{equation}
Y(s) = G(s)U(s) = \frac{K}{s(1+\tau s)} =\frac{K}{s} - \frac{K\tau}{1+\tau s}
\end{equation}

\noindent In the time domain, that is 

\begin{equation}
y(t) = K(1-e^{-t/\tau})
\end{equation}

\noindent The steady-state value is $K$ and the final value theorem applies : 

\begin{equation}
y_{ss} = \lim_{y \rightarrow \infty} y(t) = K = \lim_{s \rightarrow 0} sY(s)
\end{equation}

Therefore, the time constant $\tau$ defines the time at which around 63\% of the steady-state value is attained. Indeed, if $t=\tau$ : 

\begin{equation}
y(t=\tau) = K(1-\frac{1}{e}) \approx 0.63K 
\end{equation}

\subsection{Response to a ramp input}

In the case of a ramp, the input writes $u(t) = Kt$, which is in Laplace domain $U(s) = K/s^2$. That gives for the output : 

\begin{equation}
Y(s) = \frac{K}{s^2(1+\tau s)} = \frac{K}{s^2} + \frac{K \tau^2}{1+\tau s} - \frac{K \tau}{s}
\end{equation}

Bringing this back to the time domain, we get 

\begin{equation}
y(t) = K \left[ t - \tau(1-e^{-t/\tau})\right]
\label{ramp time response}
\end{equation}

\subsection{Application to airfoils undergoing ramps in pitch angle}

The delayed pitch angle is defined as 

\begin{equation}
\Delta \alpha' = \Delta \alpha (1-e^{-t_c/T_\alpha})
\end{equation}

In the case of airfoils whose angle of attack describes a ramp in time, we now have a formula to adjust $T_\alpha$ so that $\alpha'(t^*)$ is equal to $\alpha_{ss}$ at $t_{ds}$. Indeed, in order to comply with Sheng criterion, we need to:

\begin{itemize}
\item make the lagged dynamic stall angle $\alpha'_{ds} = \alpha'(t=t_{ds})$ constant with respect to the reduced pitch rate $r$ of the ramp motion
\item make this constant equal to the static stall angle $\alpha_{ss}$
\end{itemize}


\noindent Remembering that the output $y$ of our first order system is $\alpha'$, the procedure is to impose $\alpha'(t_{ss}) = \alpha_{ss}$ and adjust $\tau = T_\alpha$ in Equation \eqref{ramp time response} to comply with this constraint.