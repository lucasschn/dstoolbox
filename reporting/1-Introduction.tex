\section{Context}
For the last 70 years, dynamic stall has been a complex unsolved theoretical problem of fluid dynamics that have constantly aroused interest of scientists and engineers. From as early as the 1970s, pioneers such as McCroskey \cite{mccroskey_dynamic_1976} established precise phenomenological descriptions. The helicopter aerodynamicist T.S. Beddoes made much progress to model the dynamic stall phenomenon and to understand its underlying physics. 

Beddoes came up with the first generation of his dynamic stall model in \cite{beddoes_synthesis_1976}. Based on both his own observation and Ashley's theoretical framework for aeroelasticity, this model first introduced the indicial response method and the notion of effective angle of attack. These were meant to persist up to the most modern versions of the LB model. He then presented the second generation of his work in \cite{beddoes_representation_1983}. This second iteration first gave an approach to  trailing edge separation using Kirchhoff model. Together with Leishman, they finally presented in 1989 the third generation of his dynamic stall model, the one nowadays often simply referred to  as \textit{Leishman-Beddoes model}\footnote{Even though the name in reversed order (\textit{Beddoes-Leishman}) might occur at least as frequently , in the present work this model will be referred to as \textit{Leishman-Beddoes model} so that the abreviation LB cannot be confused with the abbreviation for \textit{boundary layer}, BL.}. The  lecture of the original paper \cite{leishman_semi-empirical_1989} is however not recommended, as the nomenclature is misleading and prone to implementation mistakes. Influenced by the convenient framework of control theory, Leishman also proposed an alternative formulation using the state-space form of the third-generation model in \cite{leishman_state-space_1989}.This article also features a precious table with the time-constants values for three different airfoils. Despites its high number of parameters and the empirical origin of some of its parts, the third-generation model provided a solid way of predicting the lift, drag and pitching moment characteristics of an airfoil in high speed and Reynolds number conditions (M $>0.3$, Re $> 10^6)$) . This perfectly suited the main field of application at that time: helicopter aerodynamics. For the same reason, the airfoil shape chosen in the above-cited articles was often the symmetric NACA0012 describing a pitching or plunging motion in time. 

Dynamic stall  benefited from a renewed interest since the 2000s due to the development of wind energy, especially for vertical axis wind turbines, the VAWTs. However, in wind turbine applications, the typical airfoil shapes, motions and flow conditions differ largely from helicopters. In this context, Sheng et al. have started to look at a dynamic stall criterion more adequate to lower speeds (M $<0.3$). This criterion was further developed in \cite{sheng_improved_2007} and gave birth to a modified version of the third-generation LB model in \cite{sheng_modified_2008}. Simultaneously, the focus shifted from pitching motions to ramp-up motions, which are better suited to isolate the influence of the pitch rate. Supplementary phenomenological description was provided by Mulleners \& Raffel in their article \cite{mulleners_dynamic_2013} from 2013, supported by particle image velocimetry. 

Recent developments from \cite{tank_possibility_2017} showed that dynamic stall models still deserve further investigation of the static aerodynamic properties at moderate Reynolds number. Indeed, static properties are the base on which every dynamic stall model is built and without robust and trustworthy static lift, drag and pitching moment polars, no accurate prediction of the dynamic loads can be made. 

\section{Motivation}

In \cite{sheng_improved_2007}, Sheng et al. approximate the dynamic stall angle as a piecewise linear function of the reduced pitch rate. At the same time, the choice of the critical angle of attack for stall onset assessment seemed rather arbitrary. For these reasons, there seemed to be room for improvement in this stall onset criterion and the associated modified LB model. 
Indeed, it was already observed in the data exploited in the present study that the dynamic stall time was an exponential function of the reduced pitch rate. This observation lead to the hypothesis that the piecewise linear fit of Sheng et al. was only an approximation of what should naturally be an exponential function. Additionally, the critical angle should be retought as of an angle bearing more physical meaning than the one proposed by Sheng in \cite{sheng_modified_2008}.

\section{Goals}

The present study aims at optimizing the compromise between accuracy and lightness that has been reached in the established dynamic stall models. In order to do so, special attention has been dedicated to improve Sheng's modified version of BL model, presented in \cite{sheng_modified_2008}, in particular to the critical angle used in the prediction of stall onset and to the nature of the dependency between the dynamic stall angle and the reduced pitch rate.