\documentclass[11pt]{report}
\usepackage{Preamble} % see file Preamble.sty

\begin{document}

\begin{titlepage}
    \newgeometry{margin=3cm}
	\centering
    \includegraphics[width=0.5\linewidth]{Images/epfl.png}\\[0.25cm] 	% University Logo
    \textsc{\LARGE École Polytechnique Fédérale de Lausanne}\\ \vspace{\fill}
    \textbf{\textsc{\fontsize{28}{50}\selectfont Modelling of dynamic stall}}\\ \vspace{\fill}		
	\textsc{\LARGE Master Thesis}\\[0.4cm]
	\rule{\linewidth}{0.2 mm} \\[0.5 cm]
    Lucas Schneeberger \\[2cm] \today \\
    Supervisor : Prof. Karen Mulleners \\
    Advisor : Dr. Fatma Ayancik
\end{titlepage}
\restoregeometry

\thispagestyle{numberonly}

\begin{summary}
\section*{Abstract}
Helicopters and wind turbines promote an unsteady flow around the blades due to their rotary motion. Dynamic stall models help the designers predicting the subsequent unsteady aerodynamic loads. In this work, the 3rd-generation Leishman-Beddoes  model is applied to ramp-up motions of a flat plate airfoil at moderate Reynolds number. The results are compared to the dynamic stall model designed by Sheng et al specially for low speeds ($M<0.3$). A new stall criterion that retakes Sheng et al. formulation is introduced with extended pitch rate validity range. 
\end{summary}

\tableofcontents

\chapter{Introduction}
\subfile{1-Introduction}

\chapter{Methods}
\subfile{2-Methods}

\chapter{Results and Discussion}
\subfile{3-Results and Discussion.tex}

\chapter{Conclusion}
\subfile{4-Conclusion}

\printnomenclature
\subfile{5-Nomenclature}

\bibliography{bibliography} 
\bibliographystyle{new-aiaa}

\chapter{Annexes}
\subfile{6-Annexes}

\end{document}
