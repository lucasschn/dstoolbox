\documentclass{article}
\usepackage{amsmath}
\usepackage{graphicx}


\graphicspath{{../fig/}}
\newcommand{\alphadot}{\dot{\alpha}}

\begin{document}



The time at static stall angle $t_{ss}$ is defined as the time $t$ at which the airfoil passes the static stall angle $\alpha_{ss}$.

\begin{equation}
\alpha_{ss} = \alpha(t_{ss})
\end{equation}

\begin{equation}
t_{ds} = t_{ss} +  \Delta t_1 + \Delta t_2
\end{equation}

\begin{equation}
\alpha_{ds} = \alpha(t_{ds})
\end{equation}


\section{modified Sheng criterion}

Sheng defines the lagged angle of attack, $\alpha'$ as

\begin{equation}
\Delta \alpha' = \Delta \alpha\left[1-e^{-t_c/T_\alpha} \right]
\label{eq:alpha_lag}
\end{equation}

\noindent with $t_c=tc/U_{\infty}$, the convective time, and the stall criterion for an airfoil in ramp-up motion as 

\begin{equation}
\alpha' > \alpha_{crit}
\label{eq:stall_criterion}
\end{equation}

In Sheng's article, $\alpha_{crit}$ varies with $r$. 

\begin{equation}
\alpha_{crit} =
\begin{cases}
\alpha_{ds0}, &\quad \text{if} r \geq r_0 \\
\alpha_{ds0} + (\alpha_{ds0}-\alpha_{ss}) \frac{r}{r_0}, &\quad \text{if} r < r_0 \\
\end{cases}
\end{equation}

\noindent with $r_0$ generally around 0.01.

\subsection{Exponential fit of $\alpha_{ds}$}

In his original algorithm,  Sheng defines $\alpha_{crit}$ equal to the static stall angle $\alpha_{ss}$ and uses a constant $T_\alpha$. However, it has been noticed that the range of validity of this hypothesis is restrained to $0.01<r<0.05$. Our goal in allowing $T_\alpha$ to vary with the reduced pitch rate $r$ is to improve the range of validity of the stall prediction method. 

\begin{figure}[h]
\centering
\includegraphics[width=.7\textwidth]{Sheng/alphads_r.png} 
\caption{Evolution of the stall angle and the lagged stall angle with the reduced pitch rate for a flat plate airfoil}
\label{fig:alphads_r}
\end{figure}

It has been seen from ramp-up experiments that the dynamic stall angle increases a lot with an increase in $r$ at small $r$ and only a little with $\Delta r$ at higher $r$. It even becomes constant at high $r$, as shown in Figure \ref{fig:alphads_r}. based on this observation, we use an exponential fit to identify the dependency between the dynamic stall angle $\alpha_{ds}$ and $r$ of the form: 

\begin{equation}
\alpha_{ds}(r) = A-(A-\alpha_{ss})e^{-Br} = A(1-e^{-Br})+\alpha_{ss}e^{-Br}
\label{eq:alpha_ds_r}
\end{equation}

\noindent where the value of $A$ defines the plateau ($r \rightarrow \infty$) and the limit as $r \rightarrow 0$ is equal to $\alpha_{ss}$. $B$ is the rate of increase in between these two limits. The main remaining challenge now is to find a way to express $T_\alpha$ as a function of $r$ from these two coefficients $A$ and $B$.

\subsection{Expression for $T_\alpha(r)$}

In order to be a good generalization of Sheng's original method, the new $T_\alpha$ needs to take the value predicted by Sheng somewhere in the range of validity $0.01<r<0.05$. For the given example, this means $T_\alpha$ should pass around  4.6 in this range. 

We know from the section about first order LTI systems that if the ideal angle of attack describes a motion such that 

\begin{equation}
\alpha(t) = \alphadot(t-t_0)
\end{equation}

\noindent then the lagged angle attack will be of the form:

\begin{equation}
\alpha'(t) = \alphadot\left[t- \tau(1-e^{-t/ \tau})\right]
\end{equation}

\noindent $\tau$ being the equivalent of $T_\alpha$ but in time domain. In other words, $T_\alpha = \frac{\tau c}{2U_\infty}$.

Evaluating this function at $t=t_{ds}$ and remembering the stall criterion presented in Equation \eqref{eq:stall_criterion}, we obtain : 

\begin{equation}
\alpha'(t_{ds}) = \alphadot\left[t_{ds} - \tau(1-e^{-t_{ds} / \tau})\right] = \alpha_{ss}
\label{eq:alpha_ds_tau}
\end{equation}

This equation can be solved for $\tau$ if the static properties of the airfoil and the time of dynamic stall $t_{ds}$ are known.

\subsection{Algorithm for $r$-dependent $T_\alpha$}

\begin{enumerate}
\item An airfoil with chord $c$ and known static stall angle $\alpha_{ss}$ is chosen for the experiment. 
\item The ramp-up motion is started with a defined $\alphadot$ and $r$.
\item The stall onset angle is predicted using $r$ and Equation \eqref{eq:alpha_ds_r}. From there the time of dynamic stall $t_ds$ is predicted using $\alphadot$.
\item $T_\alpha$ is computed by solving Equation  \eqref{eq:alpha_ds_tau} for $\tau$. $\alpha'$ is computed in real time using this result. 
\item When $\alpha' > \alpha_{crit}$, the stall criterion is attained and dynamic stall can be considered to have started. 
\end{enumerate}

\end{document}

\subsection{Modified Beddoes-Leishman}

Once $\alpha'$ has been obtained through the above-described procedure, it can be used to define the delayed separation point $f'$ in Beddoes-Leishman model.   

\subsection{Switching model}

All models up to now were based on a superposition of different components to normal force. It is therefore interesting to look at the performances of a model based on a switching between submodels. The switch would be triggered by the attainment of the stall and reattachment criterion.