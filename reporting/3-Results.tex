\section{LB1989 dataset}

\subsection{Leishman-Beddoes model}

According to \cite{leishman_semi-empirical_1989}

\subsubsection{NACA 0012}

\subsubsection{NACA 23012}

\section{simcos 2008 dataset with OA209}

\subsection{Leishman-Beddoes model}

\subsection{IAG model}

According to \cite{bangga_improved_2020}

\section{SH2019 dataset with modified flatplate}

In this section, the results from Henne using a flatplate with sharp edges are presented. 

\subsection{Leishman-Beddoes model}

\cite{leishman_state-space_1989}.
This article also features a precious table with the time-constants values for three different airfoils.

Figure \ref{fig:CN_LBfiltered} shows the output of Lesihman-Beddoes model with adjusted time constants. Obvisouly, some higher order are missing but the pre-stall slope, the peak and the post-stall slope until the first secondary peak are nicely modelled. The decay to the steady-state is not so consistent with experimental evidence, but that seems to be because the steady-state load is not correctly predicted. Indeed, the average experimental load for $t_c > 50$ is around 1.15 while the predicted steady load is about 1.05. That is around 10\% relative error. 

\begin{figure}[h]
    \centering
    \includegraphics[width=0.7\textwidth]{CN_LBfiltered.png}
    \caption{Comparison between LB model prediction and filtered experimental data for $r\approx 0.02$}
    \label{fig:CN_LBfiltered}
\end{figure}

However, the separation point at steady-state stands the comparison between experimental and predicted value. As shown on Figure \ref{fig:f_convectime}, the separation location $f$ stabilizes around 0.125 for the $f_exp$, found from the experimental normal coefficient via Kirchhoff model \ref{eq:kirchhoff}, and the modelled one with a modified version of Equation \eqref{eq:seppoint_flatplate} (as well as its delayed forms).

Indeed, in order to find such a good match at $30 \deg$ for the separation location, Eq. \eqref{eq:seppoint} was modified into: 

\begin{equation}
	f(\alpha) = 
		\begin{cases}
		1-0.3\exp(\frac{\alpha-\alpha_{ss}}{S_1}), &\quad \alpha \leq \alpha_{ss}\\
		0.125+0.575\exp(\frac{\alpha_{ss}-\alpha}{S_2}). &\quad \alpha > \alpha_{ss}\\
		\end{cases}
	\label{eq:seppoint_flatplate}
\end{equation}

\noindent with $\alpha_{ss} = 13.5 \deg$.

\begin{figure}[h]
    \centering
    \includegraphics[width=10cm]{f_convectime.png}
    \caption{Comparison between experimental and Leishman-Beddoes predicted separation point for $r = 0.02$}
    \label{fig:f_convectime}
\end{figure}

\subsection{IAG model}

\subsection{Sheng LB}

Using the procedure described in Sheng's 2008 article \cite{sheng_modified_2008} and presented in Section \ref{section:sheng_criterion}, the stall of the flat plate airfoil at $Re=8.4 \cdot 10^4$ was found to be well fitted by the parameters described in Table \ref{table:linfit_flatplate}.

\begin{table}[h]
    \centering
    \begin{tabular}{|c|c|c|}
        \hline
        $T_\alpha$ (-) & $\alpha_{ds,0}$ ($\deg$) & $r_0$ (-) \\
        \hline
        1.18 & 25.17 & 0.04 \\ 
        \hline        
    \end{tabular}
    \caption{Linear fit parameters for the flat plate airfoil}
    \label{table:linfit_flatplate}
\end{table}

The resulting fit is shown on Figure \ref{fig:alpha_ds_r}. 

\begin{figure}[h]
    \centering
    \includegraphics[width = 0.7\textwidth]{alpha_ds_r.png}
    \caption{Piecewise linear fit of $\alpha_{ds}(r)$ for the flat plate airfoil ($\alpha_{ss}=13.5 \deg $)}
    \label{fig:alpha_ds_r}
\end{figure}

Sheng's version of LB model was then applied to the data. The resulting normal force coefficient is shown on Figure \ref{fig:CN_LBSheng}. Once again, despite the lack of higher order harmonics, the height and the location of the first peak and the rate of decrease after stall are adequately modelled. The steady-state is underestimated in a similar way to Figure \ref{fig:CN_LBfiltered}. The advantage of Shenh's version is that $T_p$ is not anymore to be adjusted to the particular case. Instead, the linear fit parameters are computed for all pitch rates. However, it can be seen that the adequate remaining time constants $T_f$, $T_v$ and $T_{vl}$ take completely different values from Figure \ref{fig:CN_LBfiltered}.

\begin{figure}[h]
    \centering
    \includegraphics[width=0.7\textwidth]{CN_LBSheng.png}
    \caption{Comparison between Sheng model prediction and filtered experimental data for $r = 0.02$}
    \label{fig:CN_LBSheng}
\end{figure}

By taking another pitch rate, as on Figure \ref{fig:CN_LBSheng_r026} with $r=0.26$, the main peak is too high meaning that the delay caused by the unsteadiness is overestimated. When looking at the separation point location in Fig. \ref{fig:f_LBSheng_r026}, we see that $f'$ \footnote{As $T_f=0$, $f'$ and $f''$ are identical on Fig. \ref{fig:f_LBSheng_r026}} is not simply a delayed version of $f$, since $f'$ does not reach the steady-state value of $f$. Indeed, even though $\alpha'$ is a lagged version of $\alpha$, since $\alpha_{ss}$ is replaced by $\alpha_{crit}$ in Eq. \eqref{eq:fp_sheng}, $f$ and $f'$ still do not take the same steady-state value for $\alpha=30$.

\begin{figure}[h]
    \centering
    \includegraphics[width=0.7\textwidth]{CN_LBSheng_r026.png}
    \caption{Comparison between Sheng model prediction and filtered experimental data for $r = 0.026$}
    \label{fig:CN_LBSheng_r026}
\end{figure}

\begin{figure}[h]
    \centering
    \includegraphics[width=0.7\textwidth]{f_LBSheng_r026.png}
    \caption{Comparison between experimental and Leishman-Beddoes predicted separation point for $r = 0.026$}
    \label{fig:f_LBSheng_r026}
\end{figure}


\subsection{Sheng LB with exponential fit}

Based on data SH2019 data, the resulting exponential fit of $\alpha_{ds}(r)$ is displayed on Figure \ref{fig:expfit}.

\begin{figure}[h]
    \centering
    \includegraphics[width=.7\textwidth]{Sheng/alphads_r.png} 
    \caption{Evolution of the stall angle and the lagged stall angle with the reduced pitch rate for a flat plate airfoil}
    \label{fig:expfit}
\end{figure}

Using the exponential fit model described in Section \ref{section:expfit}, the normal coefficient displayed in \ref{fig:CN_LBExpfit_r020} is found. The first peak and the decay rate are nicely modelled. IN that sense, the results are similar to Leishman-Beddoes (Fig. \ref{fig:CN_LBSheng}), but with no tuning of $T_p$. The second delay time constant, $T_f$, is set to 0 here and the vortex time constants are $T_v=2$ and $T_{vl}=3.5$.

\begin{figure}[h]
    \centering
    \includegraphics[width=0.7\textwidth]{CN_LBExpfit_r020}
    \caption{Comparison between exponential fit model prediction and filtered experimental data for $r=0.20$}
    \label{fig:CN_LBExpfit_r020}
\end{figure}

It is interesting to see if the model is robust to a change in $r$ without changing the time constants. On Figure \ref{fig:CN_LBExpfit_r026}, we can see that it is not the case. 

\begin{figure}[h]
    \centering
    \includegraphics[width=0.7\textwidth]{CN_LBExpfit_r026}
    \caption{Comparison between exponential fit model prediction and filtered experimental data for $r=0.26$}
    \label{fig:CN_LBExpfit_r026}
\end{figure}