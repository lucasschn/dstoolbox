% static characteristics 
This study reinforces the idea that a reliable static model is the base for any dynamic loads predicitions. In the case of our flat plate airfoil, Kirchhoff model was not able to capture the bump in the pre-stall regime due to the presence of the laminar separation bubble (Figure \ref{fig:CN_kirchhoff}). This led to an ambiguous definition of the constant $C_{N_\alpha}$ in Leishman-Beddoes model, both the original and Sheng's version. Choices were made that priorized the accuracy of the pre-stall prediction over the steady-state regime. 

% Beddoes-Leishman 
Leishman \& Beddoes claim that their model contains only four parameters to be tuned when the airfoil motion is changed, because the remaining parameters are deduced from the static properties of the considered airfoil at the considered Reynolds number. The static properties such as $\alpha_{ss}$ and $C_{N_\alpha}$ are not always straight-forward to obtain. They might take different values depending on the interpretation (see the six different experimental stall criteria of Sheng et al. \cite{sheng_new_2006}) and the usage (for static or dynamic models). 
A more precise definition of the implied constants should be found. An experimental repeatable procedure should be determined to find all the necessary constants. Another isssue is that a lot of constants are empirical, e.g. $A_1$, $A_2$, $b_1$, and $b_2$ in the attached-flow model and $f_{ss}$, $f_\infty$ in the trailing edge separation model, and their value is always questionable when the model is not able to match the experimental data. 

% Sheng's model 
Both the original version of the Sheng-LB model and the newly introduced exponential fit have the advantage of supressing the need of tuning of the time constants, $T_p$. Unfortunately, they were found to overestimate the delay in stall onset due to the unsteadiness. $T_f$ was set to zero in both cases and even then, the height of the primary peak was generally over-predicted. The revised version of Sheng criterion seems to give a more physical decay of the lift towards the steady-state, because the LEV effect is postponed. However, the accuracy of the stall onset defined by both Sheng and the new criterion should be compared to experimental evidence. Experimental stall onset should be assessed using the criterion developed by Mulleners \& Raffel on PIV data \cite{mulleners_onset_2010}, rather than a load-based approach such as the one employed by Sheng et al. \cite{sheng_new_2006}. That way the real added value of the revised stall onset criterion for low Mach number can be estimated. In addition, when based on the chord-wise force, the experimental stall angle is suspected to be inaccurate for high pitch rates due to vicinity of the end of the ramp at $\ang{30}$.